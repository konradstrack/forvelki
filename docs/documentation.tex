\documentclass[4paper,10pt]{article}

% \documentclass[a4paper,12pt]{article}

\usepackage{amssymb}
\usepackage[polish]{babel}
\usepackage[T1]{fontenc}
\usepackage[utf8]{inputenc}
\usepackage{polski}
\usepackage{amsmath, amstext, amsopn, amsbsy, amscd, amsxtra}
\usepackage{graphicx}
\usepackage{textcomp}
\usepackage{multirow}
\usepackage{multicol}
\usepackage{fullpage}
\usepackage{syntax}

\setlength{\headheight}{15pt}
\headsep = 10pt
\usepackage{listings}
\lstset{
   language=[x86masm]Assembler,
   basicstyle=\ttfamily,
   keywordstyle=\bfseries,
   %commentstyle=\tiny,
   numbers=left,
   numberstyle=\footnotesize,
   stepnumber=1,
   numbersep=5pt,
%   backgroundcolor=\color{white},
   showspaces=false,
   showstringspaces=false,
   showtabs=false,
%   frame=single,
%   tabsize=2,
%   captionpos=b,
   breaklines=true,
   breakatwhitespace=false,
   escapeinside={\%*}{*)}
}
\usepackage{fancyhdr}
\fancyhf{}  % clear header + footer before defining new parameters

\pagestyle{fancy}

\def \labtitle {Język Forvelki}
\def \labauthors {Szymon Gut, Konrad Strack}

\title{Teoria Kompilacji - \labtitle}
\author{\labauthors}

\lhead{\labauthors}
\rhead{\labtitle}
\rfoot{strona \thepage}

\renewcommand{\headrulewidth}{0.4pt}

\lstset{numbersep=0.5cm, numberstyle=\tiny}

\begin{document}

  \begin{titlepage}
    \begin{center}
%       \includegraphics[width=0.3\textwidth]{../logo.png}
      \vskip 1cm
      EAIiE \\ Katedra Informatyki \vskip 3cm
%     \textsc{\LARGE Teoria Kompilacji}\\[1.5cm]

      \hrule \medskip
      \huge \labtitle
 \\ \smallskip
      \normalsize
      \smallskip \hrule


    \end{center}

      \vskip 10cm
      \normalsize
      \noindent \emph{Autorzy:}\smallskip\\
		\textbf{Szymon \textsc{Gut}}\\
		\textbf{Konrad \textsc{Strack}}

    \begin{center}
      \vfill
      \today
    \end{center}

  \end{titlepage}

\setcounter{section}{0}

\section{Tutorial}

\subsection{Wprowadzenie}
Forvelki to interpretowany, silnie i dynamicznie typowany język funkcyjny o w pełni leniwym wartościowaniu.
Funkcje są wartościami pierwszej klasy i mogą być definiowane anonimowe.
Język wspiera mechanizm domknięć.
Wyposażony jest w automatyczne zarządzanie pamięcią.
Umożliwia dynamiczne tworzenie pól struktur.
Zezwala na wielokrotne przypisanie wartości do zmiennych.

\subsection{Struktura kodu źródłowego}
Kod źródłowy programu w języku Forvelki składa się z ciągu instrukcji przypisania oraz wyrażeń.
Instrukcje przypisań do zmiennych modyfikują środowisko, którego używają wyrażenia.
Wyrażenia są ewaluowane i~wypisywane na ekran.

Przykładowy kod wypisujący na ekran napis \emph{hello, world} wygląda następująco.
\begin{lstlisting}
"hello, world"
\end{lstlisting}

\subsection{Typy danych}
Język Forvelki obsługuje następujące typy danych: liczby całkowie, liczby zmiennoprzecinkowe, napisy, typ logiczny, struktury oraz identyfikatory.
Literały liczb zmiennoprzecinkowych pisane są z kropką w środku, np. \texttt{3.14}.
Napisy muszą być ujęte w podwójne cudzysłowy


\subsection{Wyrażenie warunkowe}
Wyrażenie warunkowe ma postać \texttt{if <warunek> then <wyrażenie> else <wyrażenie>}.

\subsection{Funkcje}
Funkcje anonimowe są defiowane składnią \texttt{[<lista argumentów> -> <wyrażenie>]}.
Przykładowo, funkcja obliczająca sumę dwóch argumentów wygląda następująco: \texttt{[x,y -> x+y]}.
Argumenty wywołania funkcji umieszcza się w nawiasach okrągłych.
Aby funkcja mogła być rekurencyjna, musi być nazwana.
Osiąga się to przez umieszczenie nazwy przez ciałem funkcji.
Funkcja \texttt{silnia[n -> if n <= 1 then 1 else n*silnia(n-1)]} jest przykładem 

\section{Specyfikacja gramatyki języka}

\begin{grammar}
<program>       ::= <instrukcja> <separator> <program> | <instrukcja> | $\varepsilon$

<instrukcja>    ::= <definicja> | <wyrażenie>

<definicja>     ::= <przypisanie> | <skrót-funkcji>

<przypisanie>   ::= <lwartość> `=' <wyrażenie>

<skrót-funkcji> ::= `@' <nazwa> <lambda>

<funkcja>       ::= <nazwa> <lambda> | <lambda>

<lambda>        ::= `[' <lista-arg> `->' <lista-def> <wyrażenie> `]'

<lista-arg>     ::= <nazwa> <lista-arg1> | $\varepsilon$

<lista-arg1>    ::= `,' <nazwa> <lista-arg1> | $\varepsilon$

<lista-def>     ::= <definicja> <lista-def1> | $\varepsilon$

<lista-def1>    ::= <separator> <definicja> <lista-def1> | $\varepsilon$


<separator>     ::= `\\n' | `;'

<wyrażenie>     ::= <struktura> | <funkcja> | <symbol> | <liczba> | <wyrażenie> <operator> <wyrażenie> | `(' <wyrażenie> `)' | <operator-jedn> <wyrażenie> | <napis>

<operator>      ::= `+' | `-' | `*' | `/' | `<' | `>' | `==' | `!=' | `&' | `^' | `|'

<operator-jedn> ::= `+' | `-' | `~' | `!'



<lwartość>      ::= <nazwa> <lwartość1>

<lwartość1>		::= `.' <nazwa> <lwartość1> | $\varepsilon$

<struktura>              ::= `{' <lista-przypisań-str> `}'

<przypisanie-str>        ::= <lwartość> `:' <wyrażenie>

<lista-przypisań-str>    ::= <przypisanie-str> <lista-przypisań-str1> | $\varepsilon$

<lista-przypisań-str1>   ::= `,' <przypisanie-str> <lista-przypisań-str1> | $\varepsilon$

<wywolanie-funkcji>                ::= <nazwa> `(' <lista-argumentów-wywołania> `)' | <lambda> `(' <lista wyrazen> `)'

<lista-argumentów-wywołania>       ::= <wyrażenie> <lista-argumentów-wywołania1> | $\varepsilon$

<lista-argumentów-wywołania>       ::= `,' <wyrażenie> <lista-argumentów-wywołania1> |$\varepsilon$

<cyfra>         ::= `0' | `1' | `2' | `3' | `4' | `5' | `6' | `7' | `8' | `9'

<wielki-znak>   ::= `A' | `B' | `C' | `D' | `E' | `F' | `G' | `H' | `I' | `J' | `K' | `L' | `M' | `N' | `O' | `P' | `Q' | `R' | `S' | `T' | `U' | `V' | `W' | `X' | `Y' | `Z'

<mały-znak>     ::= `a' | `b' | `c' | `d' | `e' | `f' | `g' | `h' | `i' | `j' | `k' | `l' | `m' | `n' | `o' | `p' | `q' | `r' | `s' | `t' | `u' | `v' | `w' | `x' | `y' | `z'

<znak>          ::= <mały-znak> | <wielki-znak>

<nazwa>         ::= <mały-znak> <nazwa1>

<nazwa1>        ::= <znak> <nazwa1> | $\varepsilon$

<liczba>        ::= <cyfra> <liczba1> | <cyfra> <liczba1> `.' <liczba1> | <liczba1> `.' <cyfra> <liczba1>

<liczba1>       ::= <cyfra> <liczba1> | $\varepsilon$

<symbol>        ::= <wielki-znak> <symbol1>

<symbol1>       ::= <znak> <symbol1> | $\varepsilon$

<napis>         ::= ```' <literał-łańcuchowy> ```'

\end{grammar}


\end{document}
