\documentclass[4paper,10pt]{article}

% \documentclass[a4paper,12pt]{article}

\usepackage{amssymb}
\usepackage[polish]{babel}
\usepackage[T1]{fontenc}
\usepackage[utf8]{inputenc}
\usepackage{polski}
\usepackage{amsmath, amstext, amsopn, amsbsy, amscd, amsxtra}
\usepackage{graphicx}
\usepackage{textcomp}
\usepackage{multirow}
\usepackage{multicol}
\usepackage{fullpage}
\usepackage{syntax}

\setlength{\headheight}{15pt}
\headsep = 10pt
\usepackage{listings}
\lstset{
   language=[x86masm]Assembler,
   basicstyle=\ttfamily,
   keywordstyle=\bfseries,
   %commentstyle=\tiny,
   numbers=left,
   numberstyle=\footnotesize,
   stepnumber=1,
   numbersep=5pt,
%   backgroundcolor=\color{white},
   showspaces=false,
   showstringspaces=false,
   showtabs=false,
%   frame=single,
%   tabsize=2,
%   captionpos=b,
   breaklines=true,
   breakatwhitespace=false,
   escapeinside={\%*}{*)}
}
\usepackage{fancyhdr}
\fancyhf{}  % clear header + footer before defining new parameters

\pagestyle{fancy}

\def \labtitle {Język Forvelki}
\def \labauthors {Szymon Gut, Konrad Strack}

\title{Teoria Kompilacji - \labtitle}
\author{\labauthors}

\lhead{\labauthors}
\rhead{\labtitle}
\rfoot{strona \thepage}

\renewcommand{\headrulewidth}{0.4pt}

\lstset{numbersep=0.5cm, numberstyle=\tiny}

\begin{document}

  \begin{titlepage}
    \begin{center}
%       \includegraphics[width=0.3\textwidth]{../logo.png}
      \vskip 1cm
      EAIiE \\ Katedra Informatyki \vskip 3cm
%     \textsc{\LARGE Teoria Kompilacji}\\[1.5cm]

      \hrule \medskip
      \huge \labtitle
 \\ \smallskip
      \normalsize
      \smallskip \hrule


    \end{center}

      \vskip 10cm
      \normalsize
      \noindent \emph{Autorzy:}\smallskip\\
		\textbf{Szymon \textsc{Gut}}\\
		\textbf{Konrad \textsc{Strack}}

    \begin{center}
      \vfill
      \today
    \end{center}

  \end{titlepage}

\setcounter{section}{0}

\section{Tutorial}

\subsection{Wprowadzenie}
Forvelki to interpretowany, silnie i dynamicznie typowany język funkcyjny o w pełni leniwym wartościowaniu.
Funkcje są wartościami pierwszej klasy i mogą być definiowane anonimowe.
Język wspiera mechanizm domknięć.
Wyposażony jest w automatyczne zarządzanie pamięcią.
Umożliwia dynamiczne tworzenie pól struktur.
Zezwala na wielokrotne przypisanie wartości do zmiennych.

\subsection{Struktura kodu źródłowego}
Kod źródłowy programu w języku Forvelki składa się z ciągu instrukcji przypisania oraz wyrażeń.
Instrukcje przypisań do zmiennych modyfikują środowisko, którego używają wyrażenia.
Wyrażenia są ewaluowane i~wypisywane na ekran.

Przykładowy kod wypisujący na ekran napis \emph{hello, world} wygląda następująco.
\begin{lstlisting}
"hello, world"
\end{lstlisting}

\subsection{Typy danych}
Język Forvelki obsługuje następujące typy danych: liczby całkowie, liczby zmiennoprzecinkowe, napisy, typ logiczny, struktury oraz identyfikatory.
Literały liczb zmiennoprzecinkowych pisane są z kropką w środku, np. \texttt{3.14}.
Napisy muszą być ujęte w podwójne cudzysłowy


\subsection{Wyrażenie warunkowe}
Wyrażenie warunkowe ma postać \texttt{if <warunek> then <wyrażenie> else <wyrażenie>}.

\subsection{Funkcje}
Funkcje anonimowe są defiowane składnią \texttt{[<lista argumentów> -> <wyrażenie>]}.
Przykładowo, funkcja obliczająca sumę dwóch argumentów wygląda następująco: \texttt{[x,y -> x+y]}.
Argumenty wywołania funkcji umieszcza się w nawiasach okrągłych.
Aby funkcja mogła być rekurencyjna, musi być nazwana.
Osiąga się to przez umieszczenie nazwy przez ciałem funkcji.
Funkcja \texttt{silnia[n -> if n <= 1 then 1 else n*silnia(n-1)]} jest przykładem 

\section{Specyfikacja gramatyki języka}

\begin{grammar}
 <statement> ::= <ident> `=' <expr>
	\alt `for' <ident> `=' <expr> `to' <expr> `do' <statement>
	\alt `{' <stat-list> `}'
	\alt <empty> 
\end{grammar}


\end{document}
