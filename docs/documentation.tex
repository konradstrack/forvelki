\documentclass[4paper,10pt]{article}

% \documentclass[a4paper,12pt]{article}

\usepackage{amssymb}
\usepackage[polish]{babel}
\usepackage[T1]{fontenc}
\usepackage[utf8]{inputenc}
\usepackage{polski}
\usepackage{amsmath, amstext, amsopn, amsbsy, amscd, amsxtra}
\usepackage{graphicx}
\usepackage{textcomp}
\usepackage{multirow}
\usepackage{multicol}
\usepackage{fullpage}
\usepackage{syntax}

\setlength{\headheight}{15pt}
\headsep = 10pt
\usepackage{listings}
\lstset{
   language=[x86masm]Assembler,
   basicstyle=\ttfamily,
   keywordstyle=\bfseries,
   %commentstyle=\tiny,
   numbers=left,
   numberstyle=\footnotesize,
   stepnumber=1,
   numbersep=5pt,
%   backgroundcolor=\color{white},
   showspaces=false,
   showstringspaces=false,
   showtabs=false,
%   frame=single,
%   tabsize=2,
%   captionpos=b,
   breaklines=true,
   breakatwhitespace=false,
   escapeinside={\%*}{*)}
}
\usepackage{fancyhdr}
\fancyhf{}  % clear header + footer before defining new parameters

\pagestyle{fancy}

\def \labtitle {Język Forvelki}
\def \labauthors {Szymon Gut, Konrad Strack}

\title{Teoria Kompilacji - \labtitle}
\author{\labauthors}

\lhead{\labauthors}
\rhead{\labtitle}
\rfoot{strona \thepage}

\renewcommand{\headrulewidth}{0.4pt}

\lstset{numbersep=0.5cm, numberstyle=\tiny}

\begin{document}

  \begin{titlepage}
    \begin{center}
%       \includegraphics[width=0.3\textwidth]{../logo.png}
      \vskip 1cm
      EAIiE \\ Katedra Informatyki \vskip 3cm
%     \textsc{\LARGE Teoria Kompilacji}\\[1.5cm]

      \hrule \medskip
      \huge \labtitle
 \\ \smallskip
      \normalsize
      \smallskip \hrule


    \end{center}

      \vskip 10cm
      \normalsize
      \noindent \emph{Autorzy:}\smallskip\\
		\textbf{Szymon \textsc{Gut}}\\
		\textbf{Konrad \textsc{Strack}}

    \begin{center}
      \vfill
      \today
    \end{center}

  \end{titlepage}

\setcounter{section}{0}

\section{Tutorial}

\subsection{Wprowadzenie}
Forvelki to interpretowany, silnie i dynamicznie typowany język funkcyjny o w pełni leniwym wartościowaniu.
Funkcje są wartościami pierwszej klasy i istnieje możliwość definiowania funkcji anonimowych.
Język wspiera mechanizm domknięć.
Wyposażony jest w automatyczne zarządzanie pamięcią.
Umożliwia dynamiczne tworzenie pól struktur.
Zezwala na wielokrotne przypisanie wartości do zmiennych.

Na kształt języka Forvelki miały wpływ inne języki takie jak: Python, Ruby, OCaml i Haskell.

Przykładowy kod wypisujący na ekran napis \emph{hello, world} wygląda następująco.
\begin{lstlisting}
"hello, world"
\end{lstlisting}


\subsection{Struktura kodu źródłowego}
Kod źródłowy programu w języku Forvelki składa się z ciągu instrukcji przypisania oraz wyrażeń rozdzielonych literaami nowej linii bądź średnikami.
Wyrażenia są ewaluowane i~wypisywane na ekran.
Instrukcje przypisań do zmiennych modyfikują środowisko, którego używają wyrażenia.
Mają postać \texttt{<zmienna>~=~<wartość>}.
Zmienne są słowami pisanymi z małej litery.
Związane zmienne mogą mieć ponownie przypisaną wartość.


\subsection{Typy danych}
Język Forvelki obsługuje następujące typy danych: liczby całkowie, liczby zmiennoprzecinkowe, znaki, napisy, typ logiczny, struktury oraz identyfikatory.
Literały liczb zmiennoprzecinkowych pisane są z kropką w środku, np.~\texttt{3.14}.
Identyfikatory to słowa pisane z wielkiej litery.
Przy porównaniu dwa identyfikatory są równe wtedy i tylko wtedy, gdy są takimi samymi słowami.
Z typem logicznym związane są dwa identyfikatory, \texttt{True} i \texttt{False}, które oznaczają odpowiednio prawdę i fałsz.

Struktury tworzy się składnią \texttt{\{ <nazwa pola> : <wartość> , ... \}}, np. \texttt{\{x:1, y:2\}}.
Nazwa pola jest dowolnym słowem pisanym z małej litery.
Do pól można odwoływać się przez operator kropki, np. \texttt{\{x:1,~y:2\}.y}.
Gdy zmienna jest związana ze strukturą, można modyfikować jej zawartość poprzez przypisania do pól.
Można też dodawać nowe pola, jeśli pole, do którego przypisujemy wartość jeszcze nie istnieje.

Znaki muszą być ujęte w pojedyncze cudzysłowy, np.~\texttt{'a'}
Napisy natomiast w podwójne cudzysłowy, np.~\texttt{''raz dwa trzy''}.
Napis jest reprezentowany jako struktura \texttt{\{hd:<pierwszy-znak>,~tl:<reszta-napisu>\}}, przy czym napis pusty \texttt{''''} jako identyfikator \texttt{Null}.

Język jest silnie typowany i nie ma możliwości rzutowania typów.
Na liczbach można wykonywać wszystkie operacje arytmetyczne zdefiniowane w gramatyce oraz operacje porównania.
Na typie logicznym można wykonywać operację negacji \texttt{!}.
Na wszystkich typach można dokonywać operacji porównania \texttt{==} i \texttt{!=}.

\subsection{Wyrażenie warunkowe}
Wyrażenie warunkowe ma postać:
\texttt{if <warunek> then <wyrażenie> else <wyrażenie>}.
Przy czym \texttt{<warunek>} musi być wyrażeniem typu logicznego.

\subsection{Funkcje}
Funkcje anonimowe są definiowane składnią \texttt{[<lista argumentów> -> <przypisanie>;<przypisanie>;...; <wyrażenie>]}.
Przykładowo, funkcja obliczająca sumę dwóch argumentów wygląda następująco: \texttt{[x,y~->~x+y]}.
Argumenty wywołania funkcji umieszcza się w nawiasach okrągłych.
Aby funkcja mogła być rekurencyjna, musi być nazwana.
Osiąga się to przez umieszczenie nazwy przed ciałem funkcji.
Nazwa jest pojedynczym słowem pisanym z małej litery.
Funkcja obliczająca silnię argumentu może wyglądać następująco: \texttt{silnia[n -> prog=1; if n <= prog then 1 else n*silnia(n-1)]}.
Aby jednocześnie zdefiniować funkcję rekurencyjną o nazwie \texttt{<nazwa>} i przypisać ją do zmiennej o tej samej nazwie, można użyć skrótowej konstrukcji \texttt{@<nazwa>[<ciało funkcji>]}.

\subsection{Domknięcia}
Za każdym razem, gdy definiowana jest nowa funkcja, związane zmienne, których ona używa są zapamiętane.
W ten sposób, nawet gdy zostaną one później nadpisane, funkcja będzie dalej korzystać ze starych wartości.

\subsection{Leniwe wartościowanie}
Wyliczane są jedynie wartości w najbardziej zewnętrznym zasięgu, to jest nie objęte ciałem żadnej funkcji oraz wartości pośrednie wymagane do nich.
Kolejność wyliczania wartości w obrębie funkcji jest zgodna z~kolejnością przypisania do zmiennych.


\subsection{Wejście/wyjście}
Dostępna jest funkcja \texttt{write[x->x]}, która ma efekt uboczny - wypisanie argumentu na standardowe wyjście.
Funkcja \texttt{read} wczytuje wartość ze standardowego wejścia i zwraca ją.

\section{Specyfikacja gramatyki języka}

\begin{grammar}
<program>       ::= <instrukcja> <separator> <program> | <instrukcja>

<separator>     ::= `\\n' | `;'

<instrukcja>    ::= <definicja> | <wyrażenie> | $\varepsilon$

<definicja>     ::= <przypisanie> | <skrót-funkcji>

<przypisanie>   ::= <lwartość> `=' <wyrażenie>

<skrót-funkcji> ::= `@' <nazwa> <lambda>

\vskip 0.5cm

<funkcja>       ::= <nazwa> <lambda> | <lambda>

<lambda>        ::= `[' <lista-arg> `->' <lista-def> <wyrażenie> `]'

<lista-arg>     ::= <nazwa> <lista-arg1> | $\varepsilon$

<lista-arg1>    ::= `,' <nazwa> <lista-arg1> | $\varepsilon$

<lista-def>     ::= <definicja> <lista-def1> | $\varepsilon$

<lista-def1>    ::= <separator> <definicja> <lista-def1> | $\varepsilon$

\vskip 0.5cm

<wyrażenie>     ::= <wyrażenie> <operator> <wyrażenie> | `(' <wyrażenie> `)' | <operator-jedn> <wyrażenie> | <wyrażenie> `.' <lwartość> | <instrukcja-warunkowa> | <struktura> | <funkcja> | <napis> | <symbol> | <liczba>

<operator>      ::= `+' | `-' | `*' | `/' | `<' | `>' | `==' | `!=' | `<=' | `>='

<operator-jedn> ::= `+' | `-' | `!'

\vskip 0.5cm

<instrukcja-warunkowa> ::= `if' <wyrażenie> `then' <wyrażenie> `else' <wyrażenie>

\vskip 0.5cm

<lwartość>      ::= <nazwa> <lwartość1>

<lwartość1>		::= `.' <nazwa> <lwartość1> | $\varepsilon$

<struktura>              ::= `{' <lista-przypisań-str> `}'

<przypisanie-str>        ::= <lwartość> `:' <wyrażenie>

<lista-przypisań-str>    ::= <przypisanie-str> <lista-przypisań-str1> | $\varepsilon$

<lista-przypisań-str1>   ::= `,' <przypisanie-str> <lista-przypisań-str1> | $\varepsilon$

\vskip 0.5cm

<wywolanie-funkcji>                ::= <nazwa> `(' <lista-argumentów-wywołania> `)' | <lambda> `(' <lista wyrazen> `)'

<lista-argumentów-wywołania>       ::= <wyrażenie> <lista-argumentów-wywołania1> | $\varepsilon$

<lista-argumentów-wywołania>       ::= `,' <wyrażenie> <lista-argumentów-wywołania1> |$\varepsilon$

\vskip 0.5cm

<cyfra>         ::= `0' | `1' | `2' | `3' | `4' | `5' | `6' | `7' | `8' | `9'

<wielka-litera>   ::= `A' | `B' | `C' | `D' | `E' | `F' | `G' | `H' | `I' | `J' | `K' | `L' | `M' | `N' | `O' | `P' | `Q' | `R' | `S' | `T' | `U' | `V' | `W' | `X' | `Y' | `Z'

<mała-litera>     ::= `a' | `b' | `c' | `d' | `e' | `f' | `g' | `h' | `i' | `j' | `k' | `l' | `m' | `n' | `o' | `p' | `q' | `r' | `s' | `t' | `u' | `v' | `w' | `x' | `y' | `z'

\vskip 0.5cm

<litera>          ::= <mała-litera> | <wielka-litera>

<nazwa>         ::= <mała-litera> <nazwa1>

<nazwa1>        ::= <litera> <nazwa1> | <cyfra> <nazwa1> | $\varepsilon$

<liczba>        ::= <cyfra> <liczba1> | <cyfra> <liczba1> `.' <liczba1> | <liczba1> `.' <cyfra> <liczba1>

<liczba1>       ::= <cyfra> <liczba1> | $\varepsilon$

<symbol>        ::= <wielki-litera> <symbol1>

<symbol1>       ::= <litera> <symbol1> | $\varepsilon$

<napis>         ::= ```' <literał-łańcuchowy> ```'

<znak>			::= `\'' <znak-ascii> `\''

\end{grammar}


\end{document}
